% CRE Omega Demo Progress Report
% Comprehensive documentation of context, implementation, and current state
% Compile: pdflatex omega_demo_progress_report.tex

\documentclass[11pt,a4paper]{article}
\usepackage[utf8]{inputenc}
\usepackage[T1]{fontenc}
\usepackage{geometry}
\usepackage{hyperref}
\usepackage{amsmath,amssymb}
\usepackage{listings}
\usepackage{xcolor}
\usepackage{booktabs}
\usepackage{graphicx}
\usepackage{parskip}
\usepackage{enumitem}

\geometry{margin=2.5cm}
\hypersetup{colorlinks=true,urlcolor=blue,linkcolor=blue}

\lstset{
  basicstyle=\ttfamily\small,
  breaklines=true,
  frame=single,
  numbers=left,
  numberstyle=\tiny
}

\title{CRE Omega Demo: Progress Report and Technical Documentation}
\author{Common Runtime Environment (CRE) Project}
\date{February 2026}

\begin{document}

\maketitle
\tableofcontents
\newpage

%=============================================================================
\section{Project Overview}
%=============================================================================

\subsection{CRE: Common Runtime Environment}

CRE is an Erlang/OTP project implementing a YAWL (Yet Another Workflow Language) workflow engine with Petri Net-based patterns. Key characteristics:

\begin{itemize}
  \item \textbf{Language}: Erlang/OTP 25+ (tested through OTP 28)
  \item \textbf{Build}: rebar3
  \item \textbf{Core dependency}: \texttt{gen\_pnet} (the only OTP process maintaining state)
  \item \textbf{Design}: Joe Armstrong style---one real OTP runner, everything else pure helpers + message contracts
\end{itemize}

\subsection{Architecture}

\begin{itemize}
  \item \textbf{Single runtime component}: \texttt{gen\_pnet} maintains workflow state
  \item \textbf{Pure helpers}: \texttt{pnet\_types}, \texttt{pnet\_marking}, \texttt{pnet\_mode}, \texttt{yawl\_compile}, \texttt{yawl\_pattern\_expander}, \texttt{wf\_yawl\_executor}
  \item \textbf{Key modules}: \texttt{yawl\_engine}, \texttt{yawl\_executor}, \texttt{cre\_yawl\_patterns}, \texttt{gen\_yawl}
\end{itemize}

%=============================================================================
\section{AGI Symposium Omega Demo}
%=============================================================================

\subsection{Purpose}

The Omega demo executes the full AGI Symposium workflow with:
\begin{itemize}
  \item \textbf{20 LLM agents} simulating human committee roles (Chair, Program Chair, Reviewer, Ops Lead, Venue Lead, etc.)
  \item \textbf{43 YAWL workflow patterns} (P1--P43) composed in a single specification
  \item \textbf{Swarm Turing Test}: Can the system produce a transcript indistinguishable from human committee deliberation?
\end{itemize}

\subsection{Entry Points}

\begin{lstlisting}[language=bash,caption=Omega demo commands]
./scripts/run_agi_symposium_demo.sh --omega           # With Z.AI (ZAI_API_KEY required)
./scripts/run_agi_symposium_demo.sh --omega --dry-run # Curated responses, no LLM
\end{lstlisting}

\subsection{Specification}

The workflow is defined in \texttt{test/fixtures/agi\_symposium\_omega.yaml}:
\begin{itemize}
  \item Root net: \texttt{Symposium}
  \item Subnets: \texttt{ProgramThread}, \texttt{OpsThread}, \texttt{CommsThread}, \texttt{IncidentThread}, \texttt{SatelliteSymposium}
  \item Patterns: P42 ThreadSplit (split), P41 ThreadMerge (merge), P3 Synchronization (GoNoGo), P38 GeneralSyncMerge (CloseSymposium), plus 39 others
\end{itemize}

%=============================================================================
\section{Problem: Blocking at Zero Rounds}
%=============================================================================

\subsection{Observed Behavior}

Prior to the fix, running \texttt{./scripts/run\_agi\_symposium\_demo.sh --omega} produced:
\begin{itemize}
  \item \texttt{Final State: blocked}
  \item \texttt{Rounds: 0}
  \item No transitions fired; workflow never started
\end{itemize}

\subsection{Root Cause: Pattern Collision}

The Symposium root net has \textbf{13+ pattern instances} that expand into Petri net structures. When merged via \texttt{merge\_net\_structures}, internal transition and place names \textbf{collided}:

\begin{table}[h]
\centering
\caption{Transition name collisions across patterns}
\begin{tabular}{lll}
\toprule
Pattern & Transitions & User-facing rename \\
\midrule
P42 ThreadSplit & \texttt{t\_split}, \texttt{t\_finish1..4} & \texttt{t\_split} $\to$ \texttt{t\_SplitMegaThreads} \\
P41 ThreadMerge & \texttt{t\_split}, \texttt{t\_complete1..4}, \texttt{t\_merge}, \texttt{t\_finish} & \texttt{t\_merge} $\to$ \texttt{t\_MergeMegaThreads} \\
P3 Synchronization & \texttt{t\_split}, \texttt{t\_complete1..3}, \texttt{t\_join}, \texttt{t\_finish} & \texttt{t\_join} $\to$ \texttt{t\_GoNoGo} \\
P38 GeneralSyncMerge & \texttt{t\_split}, \texttt{t\_complete1..3}, \texttt{t\_join}, \texttt{t\_finish} & \texttt{t\_join} $\to$ \texttt{t\_CloseSymposium} \\
\bottomrule
\end{tabular}
\end{table}

\texttt{maps:merge} uses last-writer-wins. The final preset/postset for \texttt{t\_split} came from whichever pattern was processed last, producing a structurally broken net. In the worst case, a transition consumed the initial token but produced nothing (\texttt{\#\{\}}), causing immediate deadlock.

\subsection{Place Collisions}

Shared place names (\texttt{p\_start}, \texttt{p\_end}, \texttt{p\_joined}, \texttt{p\_branch1..3}) also collided. P3 and P38 both use \texttt{p\_branch1..3}; the place mapping should produce \texttt{p\_gonogo\_branch1..3} and \texttt{p\_close\_branch1..3}, but one overwrote the other.

%=============================================================================
\section{Solution: Per-Instance Namespacing}
%=============================================================================

\subsection{Strategy}

Namespace all \textbf{internal} (non-renamed) transitions and places per pattern instance. User-facing transitions (\texttt{t\_GoNoGo}, \texttt{t\_SplitMegaThreads}) keep their names.

\subsection{Implementation}

\subsubsection{Transition Namespacing}

\texttt{add\_internal\_transition\_namespace/3} in \texttt{yawl\_pattern\_expander.erl}:
\begin{itemize}
  \item For each transition \textbf{not} in \texttt{TrsnMap} keys, add mapping: \texttt{t\_split} $\to$ \texttt{t\_pi<N>\_split}
  \item Prefix \texttt{pi<N>} derived from \texttt{erlang:phash2(IdStr)} for pattern instance id
\end{itemize}

\subsubsection{Place Namespacing}

\texttt{add\_internal\_place\_namespace/4}:
\begin{itemize}
  \item For each place \textbf{not} in \texttt{PlaceMap} keys, add mapping: \texttt{p\_branch1} $\to$ \texttt{p\_pi<N>\_branch1}
  \item \textbf{Exception}: \texttt{p\_start} for the \textbf{entry-owning} pattern stays unnamespaced so \texttt{init\_marking} finds it
\end{itemize}

\subsubsection{Entry Owner Detection}

\texttt{compute\_entry\_owner\_id/3} in \texttt{yawl\_compile.erl}:
\begin{itemize}
  \item Uses \texttt{wf\_yaml\_spec:net\_first\_flow\_target/2} to get first flow target from input condition
  \item For Symposium: \texttt{Start $\to$ SplitMegaThreads}
  \item Pattern instance with \texttt{split\_task: SplitMegaThreads} (P42) is the entry owner
\end{itemize}

\subsubsection{Subnet Wiring Fix}

\texttt{omega\_demo\_runner} expects \texttt{p\_gonogo\_branch1..3} for P3 sync. Added \texttt{join\_task\_to\_suffix(GoNoGo) $\to$ <<"gonogo">>} so P3 produces \texttt{p\_gonogo\_branch1..3} instead of \texttt{p\_GoNoGo\_branch1..3}.

%=============================================================================
\section{Files Modified}
%=============================================================================

\begin{table}[h]
\centering
\caption{Summary of code changes}
\begin{tabular}{ll}
\toprule
File & Changes \\
\midrule
\texttt{src/core/yawl\_pattern\_expander.erl} & \texttt{add\_internal\_transition\_namespace}, \texttt{add\_internal\_place\_namespace}, \texttt{namespace\_prefix}, \texttt{is\_entry\_owner}, \texttt{join\_task\_to\_suffix} (GoNoGo clause) \\
\texttt{src/core/yawl\_compile.erl} & \texttt{compute\_entry\_owner\_id}, \texttt{pattern\_net\_matches}, pass \texttt{entry\_owner\_id} in Context \\
\texttt{src/wf/wf\_yaml\_spec.erl} & \texttt{net\_first\_flow\_target/2}, \texttt{input\_cond\_to\_atom/1} \\
\bottomrule
\end{tabular}
\end{table}

%=============================================================================
\section{Current Progress}
%=============================================================================

\subsection{Results}

\begin{table}[h]
\centering
\caption{Omega demo execution before and after fix}
\begin{tabular}{lcc}
\toprule
Metric & Before & After \\
\midrule
Rounds & 0 & 3 \\
Final State & blocked & blocked \\
Initial token consumed & No & Yes \\
Transitions fired & 0 & $\geq$3 \\
\bottomrule
\end{tabular}
\end{table}

\subsection{Namespacing Verification}

The compiled \texttt{yawl\_Symposium} module now contains:
\begin{itemize}
  \item \texttt{t\_pi36722154\_complete1..3}, \texttt{t\_pi103154468\_complete1..3} (namespaced internal transitions)
  \item \texttt{p\_gonogo\_branch1..3}, \texttt{p\_close\_branch1..3} (correct P3/P38 place names)
  \item \texttt{p\_pi29137630\_branch1..3} (namespaced internal places)
\end{itemize}

\subsection{Remaining Issue: Marking Cycle}

The demo logs: \texttt{gen\_yawl halt: marking cycle detected (fingerprint=57720917)}. This indicates:
\begin{itemize}
  \item The workflow reaches a marking that has been seen before (cycle detection in \texttt{gen\_yawl} continue handler)
  \item Execution halts to prevent infinite loops
  \item ``Blocked'' after 3 rounds is due to cycle detection, not necessarily absence of human tasks
\end{itemize}

%=============================================================================
\section{Thesis Interpretation: What Three Rounds Mean}
%=============================================================================

\subsection{Minimal Witness of Compositional Execution}

Three rounds represent the \textbf{first non-trivial witness} that the workflow engine executes composed patterns correctly:
\begin{enumerate}
  \item \textbf{Round 0}: No execution; specification structurally broken
  \item \textbf{Rounds 1--3}: Automated splits (P42), subnet runs (ProgramThread, OpsThread, CommsThread), and/or merge transitions fire
  \item \textbf{Full run}: All 43 patterns exercised; Swarm Turing Test potentially passable
\end{enumerate}

\subsection{Epistemic Significance}

The fix (namespacing) was not ``adding features'' but \textbf{making the specification's semantics consistent with its intended execution}. Three rounds demonstrate that:
\begin{itemize}
  \item The compiled net is executable
  \item Multiple patterns can coexist on the same net without semantic collision
  \item The boundary between ``broken'' and ``partially working'' has been crossed
\end{itemize}

\subsection{Swarm Turing Test}

The Swarm Turing Test asks: can the system produce a transcript indistinguishable from human committee deliberation? Passing it would demonstrate \textbf{framework insufficiency}---that fixed policy spaces cannot express certain workflow-induced behaviors. Three rounds do not pass the test but establish that the engine can \textbf{run} the specification far enough to make the question meaningful.

%=============================================================================
\section{Commands and Verification}
%=============================================================================

\subsection{Run Omega Demo}

\begin{lstlisting}[language=bash]
# With Z.AI (requires ZAI_API_KEY)
ZAI_API_KEY=your_key ./scripts/run_agi_symposium_demo.sh --omega

# Dry run (curated responses, no LLM)
./scripts/run_agi_symposium_demo.sh --omega --dry-run
\end{lstlisting}

\subsection{Compile Omega to Files}

\begin{lstlisting}[language=bash,breaklines=true]
erl -pa _build/test/lib/cre/ebin _build/default/lib/*/ebin \
    -pa _build/test/lib/yamerl/ebin -noshell -eval \
  '{ok,S}=wf_yaml_spec:from_yaml_file("test/fixtures/agi_symposium_omega.yaml"),
   {ok,_}=yawl_compile:compile_to_file(S,#{},"/tmp/omega_compiled_dump"),halt(0).'
\end{lstlisting}

\subsection{Test Suites}

\begin{lstlisting}[language=bash]
rebar3 eunit --module=yawl_compile   # 13 tests, 0 failures
rebar3 ct --suite=agi_symposium_simulation_SUITE
\end{lstlisting}

%=============================================================================
\section{Round Completion Fixes (February 2026)}
%=============================================================================

\subsection{Cycle Detection}

\texttt{gen\_yawl} halted at 3 rounds with ``marking cycle detected''. The async \texttt{continue} loop detects repeated marking fingerprints and stops to prevent infinite loops.

\textbf{Change}: Add \texttt{max\_marking\_history} and \texttt{max\_continue} to \texttt{gen\_yawl} init Options. When \texttt{max\_marking\_history = 0}, cycle detection is disabled. Omega demo passes \texttt{[\{max\_marking\_history, 0\}]} via \texttt{start\_workflow(Executor, \#{}, \#{gen\_yawl\_options => [...]})}.

\subsection{Auto-Continue (Root Net)}

\texttt{init/1} called \texttt{continue(self())}, so the async loop ran before the first \texttt{step()} from the demo runner. That consumed transitions and left 0 rounds from the user's perspective.

\textbf{Change}: Add \texttt{auto\_continue} option. When \texttt{false}, skip the initial \texttt{continue(self())}. Omega uses \texttt{[\{auto\_continue, false\}]} so \texttt{step()} drives the root net.

\subsection{Subnet Human Tasks}

Subnets (ProgramThread, OpsThread, etc.) have human tasks. Plain \texttt{drain/2} aborts at the first human task and never reaches the exit, so tokens never inject into \texttt{p\_gonogo\_branch1..3}.

\textbf{Change}: \texttt{drain\_subnet\_with\_human\_tasks/3} loops: on \texttt{abort}, find inject place via \texttt{subnet\_find\_inject\_place/2}, inject with \texttt{default\_agent} (accept), continue. Subnets can complete and inject tokens into the root for P3 sync.

\subsection{Result}

With cycle detection disabled, root step-driven, and subnet human-task auto-completion: the demo consistently reaches 3 rounds. Further progress requires subnet completion and correct P3 sync wiring.

%=============================================================================
\section{Summary}
%=============================================================================

\begin{itemize}
  \item \textbf{Problem}: Omega demo blocked at 0 rounds due to pattern transition/place name collisions when merging 13+ pattern instances on the Symposium net
  \item \textbf{Solution}: Per-instance namespacing of internal transitions and places; entry owner keeps \texttt{p\_start}; \texttt{p\_gonogo\_branch1..3} wiring fix for \texttt{omega\_demo\_runner}
  \item \textbf{Cycle \& round fixes}: \texttt{max\_marking\_history=0}, \texttt{auto\_continue=false}, \texttt{drain\_subnet\_with\_human\_tasks} for subnet completion
  \item \textbf{Outcome}: 0 $\to$ 3 rounds; workflow is executable; subnet human tasks auto-complete
  \item \textbf{Next}: Debug blocking point after 3 rounds (P3 sync or subnet wiring)
\end{itemize}

\end{document}
