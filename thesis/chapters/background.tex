% -*- latex -*-
%%
% Chapter 2: Background and Related Work
%%
%%====================================================================

\chapter{Background and Related Work}

\section{YAWL: Yet Another Workflow Language}

YAWL (Yet Another Workflow Language) was developed by van der Aalst and
ter Hofstede \cite{vandalAalst2005yawl} to address limitations in
existing workflow languages. YAWL's design is based on several key
principles:

\begin{enumerate}
  \item \textbf{Direct Petri net mapping}: YAWL workflows can be
        directly translated to Petri nets, enabling formal analysis
  \item \textbf{Pattern completeness}: YAWL supports all 20 control-flow
        patterns from the workflow patterns catalog
  \item \textbf{Orthogonal aspects}: Control flow, data flow, and resource
        perspectives are handled separately
  \item \textbf{XML-based notation}: YAWL specifications use XML for
        interoperability
\end{enumerate}

\subsection{YAWL Workflow Control Patterns}

The workflow patterns catalog \cite{vandalAalst2003workflow} identifies
20 control-flow patterns organized into several categories:

\textbf{Basic Control Flow Patterns} (WCP-01 to WCP-06):
\begin{itemize}
  \item WCP-01: Sequence
  \item WCP-02: Parallel Split
  \item WCP-03: Synchronization
  \item WCP-04: Exclusive Choice
  \item WCP-05: Simple Merge
  \item WCP-06: Multi-Choice
\end{itemize}

\textbf{Advanced Branching and Synchronization} (WCP-07 to WCP-10):
\begin{itemize}
  \item WCP-07: Synchronizing Merge
  \item WCP-08: Multi-Merge
  \item WCP-09: Discriminator
  \item WCP-10: Arbitrary Cycles
\end{itemize}

\textbf{Structural Patterns} (WCP-11 to WCP-13):
\begin{itemize}
  \item WCP-11: Implicit Termination
  \item WCP-12: Multiple Instances (without synchronization)
  \item WCP-13: Multiple Instances (with a priori design time knowledge)
\end{itemize}

\textbf{State-Based Patterns} (WCP-14 to WCP-17):
\begin{itemize}
  \item WCP-14: Multiple Instances (with a posteriori runtime knowledge)
  \item WCP-15: Deferred Choice
  \item WCP-16: Interleaved Parallel Routing
  \item WCP-17: Milestone
\end{itemize}

\textbf{Cancellation Patterns} (WCP-18 to WCP-20):
\begin{itemize}
  \item WCP-18: Cancel Task
  \item WCP-19: Cancel Case
  \item WCP-20: Cancel Region
\end{itemize}

\section{Petri Net Formalism}

Petri nets, introduced by Carl Adam Petri in 1962, provide a mathematical
formalism for modeling distributed systems \cite{peterson1981petri}.

\subsection{Definitions}

\begin{definition}[Petri Net]
A Petri net is a triple $N = (P, T, F)$ where:
\begin{itemize}
  \item $P$ is a finite set of \textbf{places}
  \item $T$ is a finite set of \textbf{transitions}
  \item $F \subseteq (P \times T) \cup (T \times P)$ is a set of arcs
\end{itemize}
\end{definition}

\begin{definition}[Marking]
A \textbf{marking} $M: P \to \mathbb{N}$ assigns a non-negative integer
number of tokens to each place. The initial marking is denoted $M_0$.
\end{definition}

\begin{definition}[Transition Firing]
A transition $t \in T$ is \textbf{enabled} in marking $M$ if
$\forall p \in \bullet t: M(p) \geq 1$, where $\bullet t$ denotes the
input places of $t$. An enabled transition may \textbf{fire}, producing
a new marking $M'$ defined by:
\begin{equation}
M'(p) = M(p) - |F(p,t)| + |F(t,p)|
\end{equation}
\end{definition}

\subsection{Soundness Properties}

For workflow nets, van der Aalst defines several soundness properties
\cite{vandalAalst1997verification}:

\begin{definition}[Soundness]
A workflow net is \textbf{sound} if:
\begin{enumerate}
  \item For every marking reachable from $M_0$, the final marking
        $M_{end}$ is reachable
  \item $M_{end}$ is the unique marking reachable from $M_0$ with at
        least one token in the output place
  \item There are no dead transitions in $M_0$ (no transition that can
        never fire)
\end{enumerate}
\end{definition}

\section{gen\_pnet Library}

The gen\_pnet library \cite{genpnet2015} provides a generic OTP behavior
for implementing Petri nets in Erlang. Key characteristics include:

\begin{itemize}
  \item Direct implementation of Petri net semantics
  \item Callback-based transition firing
  \item Token-based state representation
  \item Support for place and transition enumeration
\end{itemize}

\subsection{gen\_pnet Callbacks}

\begin{lstlisting}[language=Erlang, caption=gen\_pnet Callback Functions]
-module(my_workflow).
-behaviour(gen_pnet).

-export([place_lst/0, trsn_lst/0, init_marking/2, preset/1,
         is_enabled/3, fire/3, trigger/3]).

place_lst() -> [p1, p2, p3].
trsn_lst() -> [t1, t2].

init_marking(p1, _UsrInfo) -> [start];
init_marking(_, _) -> [].

preset(t1) -> [p1];
preset(t2) -> [p2].

is_enabled(t1, _Mode, _UsrInfo) -> true;
is_enabled(t2, _Mode, _UsrInfo) -> true.

fire(t1, #{p1 := [start]}, UsrInfo) ->
    {produce, #{p2 => [start]}, UsrInfo};
fire(t2, #{p2 := [start]}, UsrInfo) ->
    {produce, #{p3 => [done]}, UsrInfo}.

trigger(_Place, _Token, _UsrInfo) -> pass.
\end{lstlisting}

\section{XES Standard}

The IEEE 1849-2016 XES (eXtensible Event Stream) standard
\cite{xes2016standard} defines an XML-based format for event logs that
supports process mining. XES logs capture:

\begin{itemize}
  \item \textbf{Traces}: Sequences of events for a single case
  \item \textbf{Events}: Individual occurrences with timestamps
  \item \textbf{Attributes}: Key-value pairs attached to logs, traces,
        and events
\end{itemize}

\begin{lstlisting}[language=XML, caption=XES Log Example]
<log xes.version="1.0" xes.features="nested-attributes">
  <trace>
    <event>
      <string key="concept:name" value="Ordering"/>
      <date key="time:timestamp" value="2025-02-05T10:00:00"/>
      <string key="lifecycle:transition" value="start"/>
    </event>
    <event>
      <string key="concept:name" value="Ordering"/>
      <date key="time:timestamp" value="2025-02-05T10:00:05"/>
      <string key="lifecycle:transition" value="complete"/>
    </event>
  </trace>
</log>
\end{lstlisting}

\section{Related Work}

\subsection{Workflow Pattern Analysis}

Russell et al. \cite{russell2006patterns} revised the workflow
patterns catalog, adding new patterns for data flow and resource
perspectives. This thesis focuses on control-flow patterns but
acknowledges the importance of these other perspectives.

\subsection{Formal Verification of Workflows}

van der Aalst \cite{vandalAalst1997verification} established the
theoretical foundation for workflow net verification. Our analysis
applies these soundness properties to both CRE and A2A implementations.

\subsection{Process Mining}

Process mining techniques \cite{vandalAalst2011process} rely on XES
logs to discover, analyze, and improve business processes. Both CRE
and A2A implementations produce XES logs, enabling process mining
analysis of Order Fulfillment executions.
