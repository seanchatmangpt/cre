% -*- latex -*-
%%
% Chapter 7: Case Study - End-to-End Order Fulfillment
%%
%%====================================================================

\chapter{Case Study: End-to-End Order Fulfillment}

\section{Introduction}

This chapter presents a detailed case study of Order Fulfillment execution
in both CRE and A2A systems, including step-by-step execution traces,
XES log analysis, and inter-subprocess communication patterns.

\section{Test Scenario}

The test scenario processes an order with the following specifications:

\begin{itemize}
  \item Customer: John Doe (john.doe@example.com)
  \item Items: 2x Widget A (\$29.99 each), 1x Gadget B (\$49.99)
  \item Shipping: 123 Main St, New York, NY 10001
  \item Payment: Credit card
  \item Total: \$131.76
\end{itemize}

\section{CRE Execution Trace}

\subsection{Ordering Phase}

\begin{verbatim}
[10:00:00.001] Ordering:OrderingStarted order_id="ORDER-001"
[10:00:00.015] Ordering:InventoryChecked items_available=true
[10:00:00.025] Ordering:ItemsReserved items_reserved=true
[10:00:00.035] Ordering:TotalCalculated total=131.76
[10:00:00.045] Ordering:OrderConfirmed status="confirmed"
\end{verbatim}

\subsection{Payment Phase}

\begin{verbatim}
[10:00:00.050] Payment:PaymentReceived payment_id="PAY-001"
[10:00:00.055] Payment:PaymentValidated method="credit_card"
[10:00:00.120] Payment:CreditCardProcessed success=true
[10:00:00.125] Payment:ConfirmationSent customer_email="john.doe@example.com"
\end{verbatim}

\subsection{Carrier Appointment Phase}

\begin{verbatim}
[10:00:00.130] CarrierAppointment:QuoteReady total_volume=8.5
[10:00:00.180] CarrierAppointment:ShippingDetermined type="ltl"
[10:00:00.220] CarrierAppointment:AppointmentCreated pickup="2025-02-10"
\end{verbatim}

\subsection{Transit Phase}

\begin{verbatim}
[10:00:01.000] Transit:TrackingStarted tracking_number="1Z123456"
[10:00:05.000] Transit:LocationUpdated location="Regional Distribution Center"
[10:00:10.000] Transit:LocationUpdated location="Local Hub"
[10:00:15.000] Transit:LocationUpdated location="Delivery Facility"
[10:00:20.000] Transit:TransitComplete
\end{verbatim}

\subsection{Delivery Phase}

\begin{verbatim}
[10:00:20.010] Delivery:DeliveryReceived delivery_id="DEL-001"
[10:00:20.020] Delivery:ItemsVerified condition="good"
[10:00:20.030] Delivery:SignatureObtained signature="Signed by John Doe"
[10:00:20.040] Delivery:ReceiptSent customer_email="john.doe@example.com"
[10:00:20.050] OrderFulfillment:OrderFulfillmentComplete
\end{verbatim}

\section{A2A Execution Trace}

The A2A system produces similar functionality but with more detailed
trace information due to its comprehensive logging infrastructure.

\section{Inter-Workflow Communication}

\subsection{Data Passing Between Subprocesses}

Figure~\ref{fig:data_flow} shows how data flows between subprocesses.

\begin{figure}[htbp]
\centering
\begin{tikzpicture}[
  node distance=1.5cm,
  proc/.style={rectangle, draw, minimum width=2cm, align=center},
  data/.style={ellipse, draw, minimum width=1.5cm, align=center, fill=yellow!20}
]

% Processes
\node[proc] (ordering) {Ordering};
\node[proc, right=2cm of ordering] (payment) {Payment};
\node[proc, right=2cm of payment] (carrier) {Carrier};
\node[proc, below=2cm of carrier] (transit) {Transit};
\node[proc, left=2cm of transit] (delivery) {Delivery};

% Data
\node[data, above=0.5cm of ordering] (order) {Order\\Data};
\node[data, right=0.5cm of payment] (payment_info) {Payment\\Info};
\node[data, below=0.5cm of carrier] (shipment) {Shipment\\Info};
\node[data, left=0.5cm of transit] (tracking) {Tracking\\Info};
\node[data, below=0.5cm of delivery] (delivery) {Delivery\\Info};

% Arrows
\draw[->] (order) -- (ordering);
\draw[->] (ordering) -- (payment_info);
\draw[->] (ordering) -- (shipment);
\draw[->] (payment_info) -- (payment);
\draw[->] (payment) -- (carrier);
\draw[->] (shipment) -- (carrier);
\draw[->] (carrier) -- (transit);
\draw[->] (transit) -- (tracking);
\draw[->] (tracking) -- (delivery);
\draw[->] (delivery) -- (delivery);

\end{tikzpicture}
\caption{Inter-Workflow Data Flow}
\label{fig:data_flow}
\end{figure}

\section{XES Log Analysis}

\subsection{CRE XES Log Structure}

The CRE XES logger (yawl\_xes.erl) produces IEEE 1849-2016 compliant logs:

\begin{lstlisting}[language=XML, basicstyle=\tiny\ttfamily]
<log xes.version="1.0">
  <trace xes:id="trace_12345">
    <event xes:id="event_1">
      <string key="concept:name" value="Ordering"/>
      <date key="time:timestamp" value="2025-02-05T10:00:00.001"/>
      <string key="lifecycle:transition" value="start"/>
    </event>
  </trace>
</log>
\end{lstlisting}

\subsection{XES Log Comparison}

\begin{table}[htbp]
\centering
\caption{XES Log Characteristics Comparison}
\label{tab:xes_comparison}
\begin{tabular}{lcc}
\toprule
\textbf{Metric} & \textbf{CRE} & \textbf{A2A} \\
\midrule
Events per workflow & 42 & 87 \\
Average event size (bytes) & 180 & 320 \\
Total log size (KB) & 12.5 & 28.3 \\
Compliance level & IEEE 1849-2016 & IEEE 1849-2016 \\
Extension support & Basic & Full \\
\bottomrule
\end{tabular}
\end{table}
