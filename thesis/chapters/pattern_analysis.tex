% -*- latex -*-
%%
% Chapter 5: Pattern Implementation Analysis
%%
%%====================================================================

\chapter{Pattern Implementation Analysis}

\section{Introduction}

This chapter analyzes how the CRE and A2A implementations realize
the YAWL workflow control patterns. We map each pattern to its
implementation in both systems and compare approaches.

\section{Basic Control Flow Patterns}

\subsection{WCP-01: Sequence}

\textbf{Definition}: Tasks are executed sequentially, one after another.

\textbf{CRE Implementation}: Direct place-transition chain:
\begin{lstlisting}[language=Erlang, basicstyle=\tiny\ttfamily]
place_lst() -> ['p_input', 'p_step1', 'p_step2', 'p_output'].
trsn_lst() -> ['t_step1', 't_step2'].

preset('t_step1') -> ['p_input'];
preset('t_step2') -> ['p_step1'].
\end{lstlisting}

\textbf{A2A Implementation}: Predefined sequence pattern:
\begin{lstlisting}[language=Erlang, basicstyle=\tiny\ttfamily]
create_sequence([Task1, Task2], Config).
\end{lstlisting}

\subsection{WCP-02: Parallel Split}

\textbf{Definition}: A single thread of control splits into multiple
parallel threads, which execute concurrently.

\textbf{CRE Implementation}: Multiple output places from one transition:
\begin{lstlisting}[language=Erlang, basicstyle=\tiny\ttfamily]
fire('t_parallel_split', #{'p_input' := [start]}, State) ->
    {produce, #{'p_branch1' => [start], 'p_branch2' => [start]}}.
\end{lstlisting}

\textbf{A2A Implementation}: parallel\_split pattern with join coordination.

\subsection{WCP-03: Synchronization}

\textbf{Definition}: Multiple parallel threads converge into a single
thread. Synchronization waits for all incoming branches to complete.

\textbf{CRE Implementation}: Transition with multiple input places:
\begin{lstlisting}[language=Erlang, basicstyle=\tiny\ttfamily]
preset('t_synchronize') -> ['p_branch1', 'p_branch2'].
\end{lstlisting}

\subsection{WCP-04: Exclusive Choice}

\textbf{Definition}: One branch is selected from a set of alternatives
based on data-dependent conditions.

\textbf{CRE Implementation}: Enabled transitions based on state:
\begin{lstlisting}[language=Erlang, basicstyle=\tiny\ttfamily]
is_enabled('t_process_cc', _Mode, #payment_state{payment_details = Details}) ->
    maps:get(method, Details) =:= credit_card.
\end{lstlisting}

\textbf{A2A Implementation}: exclusive\_choice pattern with condition functions.

\section{Advanced Branching Patterns}

\subsection{WCP-07: Synchronizing Merge}

\textbf{Definition}: Multiple threads converge, but only one may proceed.
All branches must be ``in progress'' before the merge completes.

\textbf{CRE Implementation}: cre\_yawl.erl provides execute\_synchronizing\_merge/3.

\subsection{WCP-09: Discriminator}

\textbf{Definition}: Multiple threads converge, triggering on the
\textbf{first} completion. Subsequent completions are ignored.

\textbf{CRE Implementation}: cre\_yawl.erl provides execute\_discriminator/3.

\subsection{WCP-10: Arbitrary Cycles}

\textbf{Definition}: A process can loop back to any previous state.

\textbf{CRE Implementation}: Direct arcs from later places to earlier places.

\section{Structural Patterns}

\subsection{WCP-11: Implicit Termination}

\textbf{Definition}: A subprocess should terminate when no work remains
and all input conditions are satisfied.

\textbf{CRE Implementation}: Handled by gen\_pnet termination semantics.

\subsection{WCP-13: Multiple Instances (Design Time)}

\textbf{Definition}: Multiple instances are created based on design-time
knowledge of the required number.

\textbf{CRE Implementation}: Multiple process spawns in orchestrator.

\section{Pattern Coverage Matrix}

Table~\ref{tab:pattern_coverage} shows the pattern coverage in both
implementations.

\begin{table}[htbp]
\centering
\caption{Workflow Pattern Coverage Comparison}
\label{tab:pattern_coverage}
\small
\begin{tabular}{clcc}
\toprule
\textbf{Pattern} & \textbf{CRE} & \textbf{A2A} \\
\midrule
\textbf{Basic Patterns} & & & \\
\quad WCP-01: Sequence & \checkmark & \checkmark \\
\quad WCP-02: Parallel Split & \checkmark & \checkmark \\
\quad WCP-03: Synchronization & \checkmark & \checkmark \\
\quad WCP-04: Exclusive Choice & \checkmark & \checkmark \\
\quad WCP-05: Simple Merge & \checkmark & \checkmark \\
\quad WCP-06: Multi-Choice & \checkmark & \checkmark \\
\midrule
\textbf{Advanced Branching} & & & \\
\quad WCP-07: Synchronizing Merge & \checkmark & \checkmark \\
\quad WCP-08: Multi-Merge & \checkmark & \checkmark \\
\quad WCP-09: Discriminator & \checkmark & \checkmark \\
\quad WCP-10: Arbitrary Cycles & \checkmark & \checkmark \\
\midrule
\textbf{Structural} & & & \\
\quad WCP-11: Implicit Termination & \checkmark & \checkmark \\
\quad WCP-12: Multiple Instances (no sync) & Partial & \checkmark \\
\quad WCP-13: Multiple Instances (design time) & \checkmark & \checkmark \\
\midrule
\textbf{State-Based} & & & \\
\quad WCP-14: Multiple Instances (runtime) & Partial & \checkmark \\
\quad WCP-15: Deferred Choice & \checkmark & \checkmark \\
\quad WCP-16: Interleaved Parallel Routing & \checkmark & \checkmark \\
\quad WCP-17: Milestone & \checkmark & \checkmark \\
\midrule
\textbf{Cancellation} & & & \\
\quad WCP-18: Cancel Task & \checkmark & \checkmark \\
\quad WCP-19: Cancel Case & Partial & \checkmark \\
\quad WCP-20: Cancel Region & Partial & \checkmark \\
\bottomrule
\end{tabular}
\end{table}

\section{Formal Verification}

\subsection{Soundness Analysis}

Both implementations produce sound workflow nets according to the
definition by van der Aalst \cite{vandalAalst1997verification}:

\begin{theorem}[Soundness of CRE Order Fulfillment]
The CRE Order Fulfillment workflow net is sound because:
\begin{enumerate}
  \item Every marking reachable from the initial marking can reach the
        final marking (p\_output with token)
  \item The final marking is unique when tokens reach p\_output
  \item All transitions are live in the initial marking
\end{enumerate}
\end{theorem}

\begin{theorem}[Soundness of A2A Order Fulfillment]
The A2A Order Fulfillment workflow net is sound due to:
\begin{enumerate}
  \item Pattern-based construction guarantees sound structure
  \item Validation in yawl\_workflow\_instance ensures soundness
  \item Runtime deadlock detection prevents unsound states
\end{enumerate}
\end{theorem}
