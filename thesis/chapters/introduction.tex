% -*- latex -*-
%%
% Chapter 1: Introduction
%%
%%====================================================================

\chapter{Introduction}

\section{Research Motivation}

Business Process Management (BPM) has emerged as a critical discipline for
organizational efficiency and process optimization. At the heart of BPM
lies the workflow management system, which coordinates the execution of
business processes according to specified control flow patterns.

The YAWL (Yet Another Workflow Language) system, developed by van der
Aalst et al., represents a comprehensive approach to workflow management
that directly maps to Petri net formalism. YAWL was designed to support
all 20 control-flow patterns identified in the workflow patterns catalog,
making it one of the most expressive workflow languages available.

This thesis focuses on the Order Fulfillment process, a well-known
business case study from CAISE 2013 \cite{conforti2013supporting}, which
involves five subprocesses: Ordering, Payment, Carrier Appointment,
Freight In Transit, and Freight Delivered. This case study is particularly
interesting because it involves risk-informed decisions and complex
inter-subprocess coordination.

\section{Problem Statement}

While YAWL provides a comprehensive theoretical foundation for workflow
management, there remains a significant gap between theory and practice:
different implementations of YAWL workflows can exhibit dramatically
different characteristics in terms of:

\begin{itemize}
  \item \textbf{Architectural approach}: Direct Petri net modeling vs.
        pattern library composition
  \item \textbf{Execution semantics}: How tokens flow through the system
  \item \textbf{Pattern coverage}: Which workflow patterns are natively
        supported
  \item \textbf{Observability}: How execution traces are captured and
        analyzed
  \item \textbf{Performance}: Execution time and memory characteristics
\end{itemize}

This thesis compares two independent implementations of the YAWL Order
Fulfillment workflow:

\begin{enumerate}
  \item \textbf{CRE Implementation}: A direct Petri net modeling approach
        using the gen\_pnet behavior
  \item \textbf{A2A Implementation}: A pattern library approach with
        comprehensive workflow management infrastructure
\end{enumerate}

Both implementations target the same business process but employ
fundamentally different design philosophies, providing a unique
opportunity for comparative analysis.

\section{Contributions}

The main contributions of this thesis are:

\begin{enumerate}
  \item \textbf{Architectural Analysis}: A detailed comparison of the
        architectural approaches used in the CRE and A2A implementations,
        highlighting their trade-offs and design decisions.

  \item \textbf{Pattern Coverage Analysis}: Systematic analysis of
        workflow pattern support in both implementations, mapping YAWL
        patterns to actual code implementations.

  \item \textbf{Formal Verification}: Analysis of Petri net soundness
        properties (soundness, liveness, boundedness) for both
        implementations.

  \item \textbf{Empirical Evaluation}: Performance benchmarks comparing
        execution times and memory usage patterns for both systems.

  \item \textbf{XES Logging Analysis}: Comparison of event logging
        capabilities and process mining compatibility.

  \item \textbf{Case Study Execution}: End-to-end execution and
        documentation of the Order Fulfillment workflow in both
        systems.
\end{enumerate}

\section{Thesis Structure}

The remainder of this thesis is organized as follows:

\textbf{Chapter 2} provides background on YAWL, Petri net formalism, and
workflow patterns. It introduces the gen\_pnet library used in both
implementations and the XES standard for event logging.

\textbf{Chapter 3} describes the Order Fulfillment domain, based on the
CAISE 2013 paper. It presents the business requirements, process
specification, and risk-informed decision aspects.

\textbf{Chapter 4} presents the architectural comparison of the CRE and
A2A implementations, analyzing their design philosophies and component
structures.

\textbf{Chapter 5} analyzes the pattern implementation coverage in
both systems, mapping YAWL workflow control patterns to actual code.

\textbf{Chapter 6} presents the empirical evaluation, including
performance benchmarks, memory usage analysis, and scalability
characteristics.

\textbf{Chapter 7} provides a detailed case study of Order Fulfillment
execution, including inter-subprocess communication and data flow
analysis.

\textbf{Chapter 8} discusses lessons learned, best practices for YAWL
implementation, and limitations of both approaches.

\textbf{Chapter 9} concludes the thesis with a summary of contributions
and directions for future research.

The appendices contain complete code listings, test results, and
sample XES logs.
