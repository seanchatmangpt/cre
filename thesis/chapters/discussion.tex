% -*- latex -*-
%%
% Chapter 8: Discussion
%%
%%====================================================================

\chapter{Discussion}

\section{Lessons Learned}

\subsection{Architectural Insights}

The comparison between CRE and A2A implementations reveals several
important lessons for YAWL system designers:

\begin{enumerate}
  \item \textbf{Direct modeling vs. Pattern libraries}: Direct Petri net
        modeling (CRE) provides clearer intent and easier debugging,
        while pattern libraries (A2A) offer better reusability.

  \item \textbf{State machine vs. Callback-based}: gen\_statem (A2A) provides
        more explicit state management than gen\_pnet callbacks (CRE).

  \item \textbf{Testing investment}: A2A's comprehensive test suite
        (84,474 lines) demonstrates the value of thorough testing
        for complex workflow systems.
\end{enumerate}

\section{Best Practices}

Based on our analysis, we recommend the following best practices
for YAWL implementations:

\subsection{For Research Prototypes}

\begin{itemize}
  \item Use direct Petri net modeling (gen\_pnet) for simplicity
  \item Implement basic XES logging for process mining
  \item Focus on pattern correctness over completeness
  \item Include unit tests for each subprocess
\end{itemize}

\subsection{For Production Systems}

\begin{itemize}
  \item Use pattern library approach for reusability
  \item Implement comprehensive monitoring and telemetry
  \item Include state space exploration for verification
  \item Provide REST API for external integration
  \item Implement comprehensive error handling and recovery
\end{itemize}

\section{Limitations}

\subsection{CRE Limitations}

\begin{itemize}
  \item Limited pattern library (patterns embedded in workflows)
  \item Basic test coverage
  \item No state space exploration or formal verification
  \item Limited REST API support
  \item Smaller developer community
\end{itemize}

\subsection{A2A Limitations}

\begin{itemize}
  \item Higher complexity and learning curve
  \item More dependencies and infrastructure
  \item Higher memory footprint per instance
  \item Overkill for simple use cases
\end{itemize}

\section{Future Work}

\subsection{For CRE Implementation}

\begin{itemize}
  \item Extract patterns into reusable library
  \item Add comprehensive test suite
  \item Implement state space exploration
  \item Add REST API layer
  \item Improve XES logging with extensions
\end{itemize}

\subsection{For A2A Implementation}

\begin{itemize}
  \item Simplify API for basic use cases
  \item Reduce memory footprint
  \item Improve documentation
  \item Add direct Petri net modeling option
  \item Optimize for single-instance scenarios
\end{itemize}

\subsection{Research Directions}

\begin{itemize}
  \item Hybrid approach: Combine direct modeling with pattern library
  \item Automatic workflow verification using model checking
  \item Machine learning for workflow optimization
  \item Cross-language workflow interoperability
  \item Real-time workflow monitoring and adaptation
\end{itemize}

\section{Applicability to Other Domains}

The Order Fulfillment case study demonstrates patterns applicable to
many domains:

\begin{itemize}
  \item \textbf{Supply chain management}: Similar multi-stage processes
  \item \textbf{Financial services}: Multi-step approval workflows
  \item \textbf{Healthcare}: Patient journey through care pathways
  \item \textbf{Manufacturing}: Production process orchestration
\end{itemize}

The architectural insights and pattern implementations from both
CRE and A2A can be applied to these domains with appropriate
adaptation.
