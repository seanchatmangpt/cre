% -*- latex -*-
%%
% Chapter 9: Conclusion
%%
%%====================================================================

\chapter{Conclusion}

\section{Summary of Contributions}

This thesis presented a comprehensive comparative analysis of two YAWL
Order Fulfillment implementations: the CRE implementation and the A2A
reference implementation. Our key contributions include:

\subsection{Architectural Analysis}

We demonstrated that the CRE implementation uses direct Petri net
modeling through gen\_pnet, resulting in approximately 3,822 lines of
workflow code across 6 modules. In contrast, the A2A implementation
employs a comprehensive pattern library approach with 43 predefined
patterns and extensive infrastructure (364,842 total lines including
84,474 lines of tests).

\subsection{Pattern Coverage}

Both implementations demonstrate excellent coverage of YAWL workflow
control patterns. The CRE system implements patterns directly within
each workflow module, while A2A provides a centralized pattern library.
Key patterns such as Sequence, Parallel Split, Synchronization,
Exclusive Choice, and Milestone are correctly implemented in both systems.

\subsection{Formal Verification}

We verified that both implementations produce sound workflow nets
according to van der Aalst's definition \cite{vandalAalst1997verification}.
The Petri net structures in both systems satisfy:
\begin{itemize}
  \item Option to complete: Every marking can reach the final marking
  \item Proper completion: The final marking is unique
  \item No dead transitions: All transitions are live in initial marking
\end{itemize}

\subsection{Empirical Evaluation}

Performance benchmarks revealed interesting trade-offs:
\begin{itemize}
  \item CRE: Lower per-instance overhead (496 KB vs 992 KB)
  \item A2A: Slightly faster execution for certain workflows
        (715ms vs 760ms for full Order Fulfillment)
  \item Both: Linear scalability with concurrent instances
  \item A2A: More detailed XES logs (2.26x larger)
\end{itemize}

\section{Theoretical Implications}

This analysis contributes to the theory and practice of workflow
management in several ways:

\subsection{Pattern Implementation Strategies}

We demonstrated two viable approaches to implementing YAWL patterns:
\begin{enumerate}
  \item \textbf{Embedded patterns}: Patterns implemented within
        workflow modules (CRE)
  \item \textbf{Library patterns}: Patterns implemented as reusable
        components (A2A)
\end{enumerate}

Both approaches are valid, with different trade-offs for research
vs. production use.

\subsection{Petri Net Practice}

We validated that gen\_pnet provides an effective foundation for
implementing YAWL workflows in Erlang, with proper support for
token-based execution and place-transition semantics.

\section{Practical Applications}

The insights from this analysis can guide:
\begin{itemize}
  \item \textbf{System selection}: Choose CRE for rapid prototyping,
        A2A for production systems
  \item \textbf{Pattern design}: Understand how patterns map to
        actual implementation
  \item \textbf{Testing strategy}: Balance test coverage with
        development speed
  \item \textbf{Observability}: Design XES logging that supports
        process mining
\end{itemize}

\section{Closing Remarks}

The YAWL Order Fulfillment case study provides a valuable
benchmark for comparing workflow system implementations. Both CRE and
A2A successfully realize the Order Fulfillment process, demonstrating
the practical applicability of YAWL and Petri net formalism to
real-world business problems.

As workflow management continues to evolve with cloud computing,
microservices, and event-driven architectures, the lessons learned
from these implementations will inform the next generation of
workflow systems.

The CRE implementation demonstrates that effective workflow systems
can be built with simplicity and direct modeling, while the A2A
implementation shows how comprehensive infrastructure can provide
enterprise-grade capabilities. Together, these implementations
advance the state of the art in workflow management system design.
