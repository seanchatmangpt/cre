% -*- latex -*-
%%
% Chapter 3: YAWL Order Fulfillment Domain
%%
%%====================================================================

\chapter{The YAWL Order Fulfillment Domain}

\section{Introduction to the Case Study}

The Order Fulfillment process was introduced by Conforti et al. at
CAISE 2013 \cite{conforti2013supporting} as a case study for
risk-informed decision making during business process execution.
This chapter describes the business domain, process specification,
and the key aspects that make it an interesting case for YAWL
implementation comparison.

\section{Business Requirements}

The Order Fulfillment process handles customer orders from initial
order placement through final delivery. The process involves five
major subprocesses, each with specific business requirements:

\begin{enumerate}
  \item \textbf{Ordering}: Receive and validate customer orders,
        check inventory availability, and reserve items

  \item \textbf{Payment}: Process payment using various methods
        (credit card, PayPal, bank transfer)

  \item \textbf{Carrier Appointment}: Arrange shipping with carriers
        based on order characteristics (FTL, LTL, single package)

  \item \textbf{Freight In Transit}: Track shipment during transit,
        monitor for delays, and notify customers of issues

  \item \textbf{Freight Delivered}: Confirm delivery, verify items,
        obtain signatures, and send receipts
\end{enumerate}

\section{Process Specification}

Figure~\ref{fig:order_fulfillment_process} shows the high-level structure
of the Order Fulfillment process.

\begin{figure}[htbp]
\centering
\begin{tikzpicture}[node distance=2cm, auto,
  place/.style={circle, draw, minimum size=10mm},
  transition/.style={rectangle, draw, minimum size=10mm}]

% Places
\node[place] (p_start) {$p_{start}$};
\node[place, below of=p_start] (p_ordering) {$p_{ordering}$};
\node[place, below of=p_ordering] (p_carrier) {$p_{carrier}$};
\node[place, below of=p_carrier] (p_payment) {$p_{payment}$};
\node[place, right=3cm of p_carrier] (p_transit) {$p_{transit}$};
\node[place, below of=p_payment] (p_delivery) {$p_{delivery}$};
\node[place, below of=p_delivery] (p_end) {$p_{end}$};

% Transitions
\node[transition, right of=p_ordering] (t_ordering) {$t_{ordering}$};
\node[transition, right of=p_carrier] (t_carrier) {$t_{carrier}$};
\node[transition, left of=p_transit] (t_transit) {$t_{transit}$};
\node[transition, right of=p_payment] (t_payment) {$t_{payment}$};
\node[transition, right of=p_delivery] (t_delivery) {$t_{delivery}$};

% Arcs
\draw[->] (p_start) -- (t_ordering);
\draw[->] (t_ordering) -- (p_ordering);
\draw[->] (p_ordering) -- (t_carrier);
\draw[->] (t_carrier) -- (p_carrier);
\draw[->] (p_carrier) -- (t_payment);
\draw[->] (p_carrier) -- (t_transit);
\draw[->] (t_payment) -- (p_payment);
\draw[->] (t_transit) -- (p_transit);
\draw[->] (p_payment) -- (t_delivery);
\draw[->] (t_transit) -- (t_delivery);
\draw[->] (p_delivery) -- (t_delivery);
\draw[->] (t_delivery) -- (p_end);

\end{tikzpicture}
\caption{Order Fulfillment Process Petri Net Structure}
\label{fig:order_fulfillment_process}
\end{figure}

\subsection{Ordering Subprocess}

The Ordering subprocess implements WCP-01 (Sequence) and WCP-26 (Critical
Section) patterns:

\begin{itemize}
  \item Receive order from customer
  \item Check inventory for all items
  \item Reserve items (critical section - prevents double-booking)
  \item Calculate total (including tax and shipping)
  \item Confirm order
\end{itemize}

\subsection{Payment Subprocess}

The Payment subprocess implements WCP-04 (Exclusive Choice) and
WCP-23 (Structured Loop) patterns:

\begin{itemize}
  \item Receive payment information
  \item Validate payment details
  \item \textbf{Exclusive Choice}: Process based on payment method
    \begin{itemize}
      \item Credit card processing
      \item PayPal processing
      \item Bank transfer
    \end{itemize}
  \item Send confirmation (if successful)
  \item Handle failure (retry if under max attempts)
\end{itemize}

\subsection{Carrier Appointment Subprocess}

The Carrier Appointment subprocess implements WCP-04 (Exclusive Choice)
and WCP-18 (Milestone) patterns:

\begin{itemize}
  \item Receive confirmation of purchase order
  \item Estimate trailer usage
  \item Prepare route guide
  \item Prepare transportation quote
  \item \textbf{Exclusive Choice}: Determine shipping type
    \begin{itemize}
      \item FTL (Full Truck Load) - for large volumes
      \item LTL (Less Than Truckload) - for multiple packages
      \item Single package - for individual items
    \end{itemize}
  \item Create shipment documentation
  \item Arrange pickup and delivery appointments
  \item Produce shipping notice
\end{itemize}

\subsection{Freight In Transit Subprocess}

The Freight In Transit subprocess implements WCP-25 (Interleaved Loop)
and WCP-16 (Deferred Choice) patterns:

\begin{itemize}
  \item Start tracking
  \item Monitor shipment during transit
  \item Update location (interleaved loop)
  \item Check for delays
  \item Notify customer if delayed (deferred choice)
  \item Complete transit phase
\end{itemize}

\subsection{Freight Delivered Subprocess}

The Freight Delivered subprocess implements WCP-18 (Milestone) pattern:

\begin{itemize}
  \item Receive delivery confirmation
  \item Verify all items (milestone - must complete before signature)
  \item Obtain customer signature
  \item Send delivery receipt
  \item Complete delivery
\end{itemize}

\section{Risk-Informed Decisions}

A key aspect of the Order Fulfillment process is the handling of
risk-informed decisions, as highlighted in the CAISE 2013 paper:

\begin{enumerate}
  \item \textbf{Payment risk}: Credit card fraud detection and retry
        strategies
  \item \textbf{Inventory risk}: Critical section prevents double-booking
        of items
  \item \textbf{Transit risk}: Delay detection and customer notification
  \item \textbf{Delivery risk}: Item verification before completion
\end{enumerate}

Both CRE and A2A implementations must handle these risk scenarios,
though their approaches may differ significantly.

\section{Data Flow Between Subprocesses}

Table~\ref{tab:subprocess_data_flow} summarizes the data flow
between subprocesses.

\begin{table}[htbp]
\centering
\caption{Data Flow Between Order Fulfillment Subprocesses}
\label{tab:subprocess_data_flow}
\begin{tabular}{lll}
\toprule
\textbf{From Subprocess} & \textbf{To Subprocess} & \textbf{Data Passed} \\
\midrule
Ordering & Payment & Order ID, total amount, customer info \\
Ordering & Carrier & Order ID, items list, shipping address \\
Carrier & Transit & Shipment ID, tracking number, carrier info \\
Payment & Transit & Payment confirmation (via orchestrator) \\
Transit & Delivery & Shipment status, delivery details \\
\bottomrule
\end{tabular}
\end{table}
